\chapter{Introdução}
\label{chap:intro}

De acordo com \citeonline{Rubio} plataformas móveis podem ser classificadas, quanto ao sistema de locomoção, como terrestres, aquáticos e aéreos. Os terrestres são subdivididos em robôs que possuem rodas, pernas (bípedes) ou esteiras, em que cada um  possui características específicas quanto ao movimento a ser realizado. Os bípedes, por exemplo, simulam um caminhar antropomórfico, semelhante aos humanos. 

De acordo com \citeonline{He2020}, mais de 50\% da superfície terrestre é inacessível por veículos tradicionais com rodas e trilhas. Enquanto, robôs com pernas possuem uma maior mobilidade em ambientes irregulares porém, eles possuem mecânismos e modos de controle muito mais complexos.\cite{Wieber20161203}

Segundo \citeonline{He2020} a perfomance de robôs com pernas esta relacionada com vários fatores, incluindo seus mecânismos, atuação, percepção, e métodos de controle. Por isso, é necessário realizar uma pesquisa sobre a evolução dessas tecnologias.

O desenvolvimento desta documentação consiste na análise dos robôs humanóides tendo como base alguns dos principais artigos relacionados ao seu estudo assim como informações disponibilizadas pelos desenvolvedores, destacando suas principais características físicas como tipos de sensores utilizados pelo sistema de percepção e estratégias de design mecânico e também informações relacionadas ao sistema de controle.

\section{Objetivo}
\label{sec:obj}

O estudo bibliográfico foi realizado com o intuito de auxiliar no desenvolvimento do projeto Walker que tem como objetivo construir um robô humanóide autônomo de pequeno porte. Esta pesquisa visa contribuir com informações obtidas através das análise das principais características dos sistemas dos robôs humanoídes já existentes. 

\section{Justificativa}
\label{sec:justi}

Robôs antropomórficos são amplamente utilizados em diversas áreas no dia-a-dia, desde interações com humanos até aplicações na área da saúde, bem como em pesquisas acadêmicas, sendo uma configuração mais adequada para transposição de ambientes de difícil navegação.

\citeonline{Gupta2017607} destacam que a vantagem da locomoção por pernas é a utilização de passos discretos para o equilíbrio e a movimentação do robô, o que permite que este realize manobras em terrenos acidentados e escadas. E, dentre os robôs com pernas, os robôs bípedes oferecem outras vantagens como mãos livres para manipulação, locomoção com maior eficiência energética e a capacidade de torcer os pés para rotacionar em torno do eixo logintudinal do seu próprio corpo.

Uma abordagem trazida por \citeonline{Joseph2018} aponta a aplicação dos robôs humanóides para auxiliar cuidadores e pacientes, especialmente em áreas de risco, como ambientes contaminados. Podendo ser utilizados para usos médicos e cirúrgicos, para dar assistência aos pacientes e cuidadores e também para a área de segurança.

E, segundo \citeonline{Chatterjee2017603} os robôs humanóides são vantajosos para serem aplicados em atividades domésticas, podendo ajudar no cuidado de idosos, podem ser uma fonte de entretenimento, além de realizar as tarefas domésticas.
