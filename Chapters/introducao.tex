\chapter{Introdução}
\label{chap:intro}

Este pode ser um parágrafo citado por alguém \cite{Barabasi2003-1} e \cite{barabasi2003linked}.
Para ajustar veja o comentário do capítulo \ref{chap:fundteor}.

As orientações do robô são em três dimensões \cite{aperea-1}.
Segundo \citeonline{aperea-1}

Segundo \citeonline{barabasi2003linked}, ...

%--------- NEW SECTION ----------------------
\section{Objetivos}
\label{sec:obj}

% \begin{justifY}
%   Este projeto consiste em desenvolver um robô bípede de pequeno porte, ou seja que se desloca sobre dois pés. O robô deve ser capaz de se locomover e desviar de obstáculos em um determinado ambiente.
% \end{justify}

Este projeto consiste em desenvolver um robô bípede de pequeno porte, ou seja que se desloca sobre dois pés. O robô deve ser capaz de se locomover e desviar de obstáculos em um determinado ambiente. 
\label{sec:obj}

%//todo incluir justify e p flushright 

% \subsection{Objetivos Específicos}
% \label{ssec:objesp}
% Os objetivos específicos deste projeto são:
% \begin{itemize}
%       \item Desenvolver habilidades de gestão de projetos.
%       \item Desenvolver algoritmos utilizando ROS;
%       \item Utilizar visão computacional;
%       \item Simular um robô no gazebo;
%   \end{itemize}

% \subsubsection*{Objetivos específicos principais}
% \label{sssec:obj-principais}
% ok vendo Aqui


\begin{equation}
\label{eq:energia}
  E=mc
\end{equation}


\begin{equation*}
  m=(\frac{E}{c})
\end{equation*}


\begin{equation*}
  m=\Bigg(\frac{E}{c}\Bigg)
\end{equation*}


\begin{equation}
  m=E/c
\end{equation}


%--------- NEW SECTION ----------------------
\section{Justificativa}
\label{sec:justi}

O pesquisador/estudante deve apresentar os aspectos mais
relevantes da pesquisa ressaltando os impactos (e.g. cient\'ifico,
tecnol\'ogico, econ\^omico, social e ambiental) que a pesquisa
causar\'a. Deve-se ter cuidado com a ingenuidade no momento em que
os argumentos forem apresentados.
De acordo com a equação \ref{eq:energia}




%--------- NEW SECTION ----------------------
\section{Organização do documento}
\label{section:organizacao}

Este documento apresenta $5$ capítulos e está estruturado da seguinte forma:

\begin{itemize}

  \item \textbf{Capítulo \ref{chap:intro} - Introdução}: Contextualiza o âmbito, no qual a pesquisa proposta está inserida. Apresenta, portanto, a definição do problema, objetivos e justificativas da pesquisa e como este \thetypeworkthree está estruturado;
  \item \textbf{Capítulo \ref{chap:fundteor} - Fundamentação Teórica}: XXX;
  \item \textbf{Capítulo \ref{chap:metod} - Materiais e Métodos}: XXX;
  \item \textbf{Capítulo \ref{chap:result} - Resultados}: XXX;
  \item \textbf{Capítulo \ref{chap:conc} - Conclusão}: Apresenta as conclusóes, contribuições e algumas sugestões de atividades de pesquisa a serem desenvolvidas no futuro.

\end{itemize}
